\documentclass[12pt,a4paper]{report}
\usepackage[top=2.5cm, bottom=2.5cm, left=2.5cm, right=2.5cm]{geometry}
\usepackage{amsmath}
\usepackage{amssymb}
\usepackage[round]{natbib}
\usepackage{graphicx}
\usepackage{tabularx}
\usepackage{algorithm}
\usepackage{algpseudocode}
\usepackage{listings}
\usepackage[usenames,dvipsnames,svgnames]{xcolor}
\usepackage{url}
\newcommand{\vect}[1]{\boldsymbol{#1}}
\usepackage{hyperref}
\hypersetup{
pdfnewwindow=true,      % links in new window
colorlinks=true,        % false: boxed links; true: colored links
linkcolor=Blue,         % color of internal links
citecolor=Blue,         % color of links to bibliography
filecolor=Blue,         % color of file links
urlcolor=Blue           % color of external links
}

\begin{document}

\begin{center}
  \huge
  {
   \vspace*{1.0cm}
   GPUQT: \\
   Graphics Processing Units Genetic Algorithm \\
   \vspace*{1.0cm}
   Reference Manual\\
   \vspace*{1.0cm}
   Version 1.0 \\
   \vspace*{1.0cm}
   (Nov 05, 2019)\\
  \vspace*{2.0cm}
  }
  \large
  {
  Author: \\
  Zheyong Fan (Aalto University)\\
  }
  \vspace*{1.0cm}
\end{center}

\tableofcontents

\chapter{Introduction\label{chapter:introduction}}

\section{What is GPUGA?}

\verb"GPUGA" stands for Graphics Processing Units Genetic Algorithm. It is an efficient empirical potential fitting code using the genetic algorithm (GA) implemented on the graphics processing units (GPU). The implementation language is CUDA C++. It is developed for this paper \cite{fan2019arxiv}:

Zheyong Fan, Yanzhou Wang, Xiaokun Gu, Ping Qian, Yanjing Su, Tapio Ala-Nissila, A minimal Tersoff potential for diamond silicon with improved descriptions of elastic and phonon transport properties, https://arxiv.org/abs/1909.11474

If you use \verb"GPUGA" in your published work, we kindly ask you to cite Ref. \cite{fan2019arxiv}.

This is version 1.0, and it can only fit the minimal Tersoff potential as proposed in Ref. \cite{fan2019arxiv}. We aim to implement more empirical potentials for version 2.0.

\section{Feedbacks}

You can email Dr. Zheyong Fan if you find errors in the manual or bugs in the source code, or have any suggestions/questions about the manual and code. The following email address can be used:
\begin{itemize}
\item brucenju(at)gmail.com
\end{itemize}


\section{Acknowledgments}
We acknowledge the computational resources provided by Aalto Science-IT project and Finland's IT Center for Science (CSC).

\chapter{Theoretical formalisms\label{section:theory}}

See the above paper. 

\chapter{Using GPUGA \label{section:usage}}

The code has only been tested in linux operating systems and we assume that the user is using a linux operating system to compile and run this code.

\section{Compile the code and run the examples}

\subsection{Compiling}

After downloading and unpacking \verb"GPUGA", one can see three folders:  \verb"src",  \verb"doc",  \verb''tools'', and \verb"examples". The folder \verb"src" contains all the source files. The folder \verb"examples" contains all the examples. The folder \verb"tools" contains a Matlab script for plotting Fig. 3 in the above paper. The folder \verb"doc" contains the pdf file you are reading and the source files generating it.

To compile the code, first go to the \verb"src" folder. The just type
\begin{verbatim}
    make -f makefile.cpu
\end{verbatim}
in the command line. This will produce an executable called \verb"gpuga" in the \verb"src" folder. The second line of \verb"makefile" reads
\begin{verbatim}
    CFLAGS = -std=c++11 -O3 -arch=sm_50 --use_fast_math
\end{verbatim}
Change \verb"gpuga" to the appropriate ``compute capability'' of your GPU. The minimum compute capability supported by \verb"GPUGA" is 3.5.

\subsection{Running}

Go to the directory where you can see the \verb"src" folder and type
\begin{verbatim}
    src/gpuga < examples/input.txt
\end{verbatim}
This will run the prepared example. To run your own calculations, just replace folder name (including absolute or relative path but excluding a ``/'') as specified in examples/input.txt to your own working directory. Then you need to prepare some input files in the working directory, as describe below.

\section{Input files for GPUGA}

To run on simulation with GPUGA, you need to prepare three files in your chosen working directory: \verb''ga.in", \verb"potential.in", and \verb"train.in". We describe them below.

\subsection{The ga.in input file}

This file contains the controlling parameters defining the GA simulation. In this input file, blank lines and lines started with \verb"#" are ignored. Each non-empty line starts with a keyword followed by one parameter.  The valid keywords and their parameters are listed below.
\begin{enumerate}
\item  \verb"maximum_generation"\\
This keyword needs one parameter, which is the maximum number of generations in the GA evolution. It can be any positive integer. The default value is 1000.
\item  \verb"population_size" \\
This keyword needs one parameter, which is the population size (number of individuals in one generation). It can be no less than 20 and should be a multiple of 10. The default value is 200.
\item  \verb"parent_number"\\
This keyword needs one parameter, which is the number of parents in one generation. It can be no less than 10 and should be a multiple of 10. The default value is 100.
\item  \verb"mutation_rate"\\
This keyword needs one parameter, which is the initial mutation rate in GA evolution. It should be in $[0,1]$. The default value is 0.2. The mutation rate will linearly decrease and reach 0 up to the \verb"maximum_generation".
\end{enumerate}

\subsection{The potential.in input file}

This file contains information about the potential to be fitted. It has a fixed format. For example, the file \verb"examples/si_diamond/potential.in" reads:
\begin{verbatim}
potential_type 1
cutoff         3.0
weight_force   0.05
weight_energy  0.15
weight_stress  0.8
D0             2.9 3.3
a              1.3 1.5
r0             2.2 2.4
S              2 2
n              0.5 0.8
beta           0 0.4
h              -0.8 -0.6
R1             2.8 2.8
R2             3.2 3.2
\end{verbatim}
At each line, there is one character string and one or two numbers. To do new experiments, just keep the strings unchanged and modify the numbers. Here are some explanations:
\begin{itemize}
\item line 1: The type of the potential to be fitted, which can only be 1  (means the minimum Tersoff potential) in this version.
\item line 2: The cutoff distance used for building the neighbor list for each configuration (see below), which should be a positive number.
\item lines 3-5: The weighting factors for force, energy, and stress. They should be non-negative numbers. It is good the make their summation to be 1, although the code does not complain if this is not the case.
\item lines 6-14: The lower (the first number in each line) and upper (the second number in each line) bounds of the potential parameters for the minimal Tersoff potential. The order of the parameters are the same as those in Table II of Ref. \cite{fan2019arxiv}.
\end{itemize}


\subsection{The train.in input file}

This file contains all the training data, possibly from DFT calculations. It's format is fixed to:
\begin{verbatim}
Nc Nc_force
Na_1
Na_2
...
Na_Nc
data for force configuration 1
data for force configuration 2
...
data for force configuration Nc_force
data for energy/virial configuration 1
data for energy/virial configuration 1
...
data forenergy/virial configuration Nc - Nc_force
\end{verbatim}



\section{Output files of GPUGA}

For each simulation, four output files will be generated in the working directory: \verb"ga.out", \verb"force.out", \verb"energy.out", and \verb"virial.out". A Matlab script \verb"plot_results.m" can be used to plot a figure similar to Fig. 2 in Ref. \cite{fan2019arxiv}.

\subsection{The ga.out file}

This file will contain \verb"maximum_generation" lines (or less if the simulation is terminated ahead of time by the user) and 11 columns. Each line contains the following items
\begin{verbatim}
step best_fitness D0 alpha r0 S n beta h R1 R2
\end{verbatim}
Here, \verb"step" is the step in the GA evolution, starting with 0, \verb"maximum_generation" is the fitness function (the smaller, the better) for the elite (the best solution) in each generation, and the remaining 8 numbers are the potential parameters (in the same order as in \verb"potential.in") for the elite in each generation.

To find the best solution for one simulation, just check the last line.

\subsection{The force.out file}

There are 6 columns. The first three columns are the $x$, $y$, and $z$ force components in the force configurations calculated from the best solution. The last three columns are the corresponding forces from \verb"train.in". The first $N_1$ rows correspond to the $N_1$ atoms in the first force configuration; the next $N_2$ rows correspond to the $N_2$ atoms in the second force configuration; and so on.

\subsection{The energy.out file}

There are 2 columns. The first column gives the energies calculated calculated from the best solution. The second column gives the corresponding energies from \verb"train.in". Each row corresponds to one energy/virial configuration \verb"train.in".

\subsection{The virial.out file}

There are 2 columns. The first column gives the virials calculated calculated from the best solution. The second column gives the corresponding virials from \verb"train.in". The number of row is a multiple of 6 and the first 1/6 corresponds to the $xx$ component of the viral in the same order as the energy data in \verb"energy.out", followed by the $yy$, $zz$, $xy$, $yz$, and $zx$ components.

\bibliographystyle{plainnat}

\bibliography{refs}


\end{document}










